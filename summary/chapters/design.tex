\section{Өгөгдлийн сангийн диаграмм}
\begin{figure}[ht]
  \centering
  \includegraphics[width = \textwidth-2.5cm]{images/dbDiagram.png}
  \caption{Өгөгдлийн сангийн диаграмм}\label{fig:dbDiagram}
\end{figure}
\newpage

\section{Өгөгдлийн элемент}
Чатбот системийн өгөгдлийн сангийн диаграммд харуулсан хүснэгтүүдэд агуулагдах мэдээлэл болон үүргийн талаар дэлгэрэнгүй тайлбарласан болно. 
\subsection{advertisement - Ажлын байрны зар}
Ажлын байрны зар нь ямар категори буюу ангилалд, ямар холбоо барих хаягийн хамтаар хадгалагдаж буй мэдээлэл болон бусад дэлгэрэнгүй мэдээллийг харуулсан байна. 
\begin{table}[ht]
  \begin{tabular}{|l|l|c|c|c|l|}
  \hline
  \multicolumn{1}{|c|}{\textbf{№}} & \multicolumn{1}{c|}{\textbf{Баганын нэр}} & \textbf{\begin{tabular}[c]{@{}c@{}}Түлхүүр \\ өгөгдөл\end{tabular}} & \textbf{\begin{tabular}[c]{@{}c@{}}Өгөгдлийн\\ төрөл\end{tabular}} & \textbf{\begin{tabular}[c]{@{}c@{}}Хоосон\\ утга\end{tabular}} & \multicolumn{1}{c|}{\textbf{Тайлбар}}                                                  \\ \hline
  1                                & \textbf{\_id}                             & PK                                                                  & varchar                                                            & not null                                                       & \begin{tabular}[c]{@{}l@{}}Ажлын байрны зарын дахин\\ давтагдашгүй дугаар\end{tabular} \\ \hline
  2                                & category\_id                              & FK                                                                  & varchar                                                            & not null                                                       & \begin{tabular}[c]{@{}l@{}}Ажлын байрны зард хамаарах\\ ангилалын дугаар\end{tabular}  \\ \hline
  3                                & url                                       &                                                                     & varchar                                                            & not null                                                       & Ажлын байрны зарын хаяг                                                                \\ \hline
  4                                & company                                   &                                                                     & varchar                                                            & not null                                                       & Ажил олгогч компани / хувь хүн                                                         \\ \hline
  5                                & title                                     &                                                                     & varchar                                                            & not null                                                       & Ажлын зарын гарчиг                                                                     \\ \hline
  6                                & roles                                     &                                                                     & varchar                                                            & null                                                           & Гүйцэтгэхүндсэн үүрэг                                                                  \\ \hline
  7                                & requirements                              &                                                                     & varchar                                                            & null                                                           & \begin{tabular}[c]{@{}l@{}}Ажлын байранд тавигдах\\ шаардлага\end{tabular}             \\ \hline
  8                                & additionalInfo                            &                                                                     & varchar                                                            & null                                                           & Нэмэлт мэдээлэл                                                                        \\ \hline
  9                                & city                                      &                                                                     & varchar                                                            & null                                                           & Ажлын байрны харъяа хот                                                                \\ \hline
  10                               & district                                  &                                                                     & varchar                                                            & null                                                           & Ажлын байрны харъяа дүүрэг                                                             \\ \hline
  11                               & exactAddress                              &                                                                     & varchar                                                            & null                                                           & Ажлын байрны бүтэн хаяг                                                                \\ \hline
  12                               & level                                     &                                                                     & level\_types                                                       & null                                                           & Ажлын түвшин                                                                           \\ \hline
  13                               & type                                      &                                                                     & workTime\_type                                                     & null                                                           & Ажиллах цагийн төрөл                                                                   \\ \hline
  \end{tabular}
  \end{table}
\begin{table}[ht]
  \caption{advertisement хүснэгт}\label{table:advertisement}
    \begin{tabular}{|l|l|c|c|c|l|}
    \hline
    \multicolumn{1}{|c|}{\textbf{№}} & \multicolumn{1}{c|}{\textbf{Баганын нэр}} & \textbf{\begin{tabular}[c]{@{}c@{}}Түлхүүр \\ өгөгдөл\end{tabular}} & \textbf{\begin{tabular}[c]{@{}c@{}}Өгөгдлийн\\ төрөл\end{tabular}} & \textbf{\begin{tabular}[c]{@{}c@{}}Хоосон\\ утга\end{tabular}} & \multicolumn{1}{c|}{\textbf{Тайлбар}} \\ \hline
    14                               & minSalary                                 &                                                                     & float                                                              & null                                                           & Доод цалин                            \\ \hline
    15                               & maxSalary                                 &                                                                     & float                                                              & null                                                           & Дээд цалин                            \\ \hline
    16                               & isDealable                                &                                                                     & boolean                                                            & null                                                           & Цалин тохиролцох эсэх                 \\ \hline
    17                               & phoneNumber                               &                                                                     & varchar                                                            & null                                                           & Холбоо барих утасны дугаар            \\ \hline
    18                               & fax                                       &                                                                     & varchar                                                            & null                                                           & Холбоо барих факсын дугаар            \\ \hline
    19                               & publishedDate                             &                                                                     & datetime                                                           & not null                                                       & Зар нийтэлсэн огноо                   \\ \hline
    \end{tabular}
  \end{table}
Энд \textit{level} буюу ажлын түвшин, \textit{type} буюу ажлын цагийн өгөгдлийн төрлийг тодорхойлохдоо дараах байдлаар зааж өгсөн.
\\\textbf{Enum level\_types} буюу ажлын түвшний шаардлага нь дараах үндсэн 4 өгөгдлийн төрлөөс хамаарна:
\begin{itemize}
  \item student - Оюутан / дадлагажигч
  \item professional - Мэргэжлийн
  \item occupasionDoesntRequire - Мэргэжил шаардахгүй
  \item intermediateManagemet - Дунд шатны удирдлага
  \item topLevelManagemet - Дээд шатны удирдлага
\end{itemize}
\textbf{workTime\_type} буюу ажиллах цагийн нөхцөл нь дараах үндсэн 4 өгөгдлийн төрлөөс хамаарна:
\begin{itemize}
  \item shift - Ээлжийн
  \item fullTime - Бүтэн цагийн
  \item halfTime - Хагас цагийн
  \item contract - Гэрээт / зөвлөх
  \item seasonal - Улирлаар
\end{itemize}

\subsection{category - Ангилал}
Ажлын байрны зарын бүх ангиллуудын хаяг болон нэрийн мэдээллийг хадгалах хүснэгт юм. Ангиллууд нь дэд ангилал байж болох учир түүнийг эцэг ангиллын дугаарыг хадгалах байдлаар зохиомжлов. 
\begin{table}[ht]
  \caption{category хүснэгт}\label{table:category}
  \begin{tabular}{|l|l|c|l|c|l|}
  \hline
  \multicolumn{1}{|c|}{\textbf{№}} & \multicolumn{1}{c|}{\textbf{Баганын нэр}} & \textbf{\begin{tabular}[c]{@{}c@{}}Түлхүүр\\ өгөгдөл\end{tabular}} & \multicolumn{1}{c|}{\textbf{\begin{tabular}[c]{@{}c@{}}Өгөгдлийн\\ төрөл\end{tabular}}} & \textbf{\begin{tabular}[c]{@{}c@{}}Хоосон\\ Утга\end{tabular}} & \multicolumn{1}{c|}{\textbf{Тайлбар}}                                   \\ \hline
  1                                & \textbf{id}                               & PK                                                                 & varchar                                                                                 & not null                                                       & \begin{tabular}[c]{@{}l@{}}Ажлын байрны зарын\\ ангиллын дугаар\end{tabular} \\ \hline
  2                                & url                                       &                                                                    & varchar                                                                                 & not null                                                       & Ангиллын хаяг                                                          \\ \hline
  3                                & name                                      &                                                                    & varchar                                                                                 & not null                                                        & Ангиллын нэр                                                           \\ \hline
  4                                & parent\_id                                & FK                                                                 & varchar                                                                                 & null                                                           & Эцэг ангиллын дугаар                                                   \\ \hline
  \end{tabular}
\end{table}

\section{Өгөгдлийн сангийн холбоосын тайлбар}
\begin{itemize}
  \item Нэг ангилал буюу категорид олон ажлын байрны зар байж болно.
  \item Нэг ангилал буюу категорид олон категори байж болно. 
  \item Нэг ажлын байрны зард нэг категори байна.
\end{itemize}

\section{Класс диаграм}
Програмын класс диаграмд классууд болон тэдгээрийн хоорондын хамаарлыг дараах байдлаар тодорхойлж үндсэн үүрэг, ажиллах зарчмуудыг тайлбарлав.
\begin{figure}[ht]
  \centering
  \includegraphics[width = \textwidth-2cm]{images/classDiagram.png}
  \caption{Класс диаграм} \label{fig:classDiagram}
\end{figure}
\subsection{Бот класс}
Энэ классын үүрэг нь үндсэн Bot-ийн тохиргоог хийх, хэрэглэгчээс ирсэн асуултыг холбогдох класс-руу дамжуулна. Өөрөөр хэлбэл чатботыг ажиллуулах үндсэн эх бие нь юм. Мөн ActivityHandler интерфэйсд байх OnMessage, OnMembersAdded функцийг хэрэгжүүлэх бөгөөд эдгээр нь шинэ хэрэглэгч орж ирсэн талаарх \textit{session}-г дамжуулдаг бөгөөд хэрэглэгчтэй харьцах цөм хэсгийг агуулна.
\subsection{QuestionUnderstand класс}
Энэхүү класс нь хэрэглэгчийн асуусан асуултыг боловсруулж түлхүүр үгийг таних үүрэгтэй. Урьдчилан бэлдсэн, хариулж чадах асуултын дагуу түлхүүр үгийг түүж аван, ямар төрлийн асуулт болохыг тодорхойлно. Улмаар үр дүнг \textit{ApiHelper} классруу дамжуулна.
\subsection{ApiHelper класс}
Энэ классын үүрэг нь орж ирсэн түлхүүр үгийг ашиглан өгөгдлийн сантай холбогдон query илгээж үр дүнг хүлээн авна. Хүлээн авсан үр дүн JSON форматын өгөгдөл байх тул CardBuilder класс руу дамжуулна.
\subsection{CardBuilder класс}
Энэхүү класс нь орж ирсэн үр дүнг буюу хэрэглэгчдэд харуулах текстийг Microsoft Botframework-ийн AdaptiveCard-ийг ашиглан хэрэглэгчдэд бүтэцлэн харуулах үүрэгтэй.
