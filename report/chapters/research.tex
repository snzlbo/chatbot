\section{Системийн судалгаа}
Сонгосон сэдэв болох ”Ажил олгогчдын өгөгдлийн анализ систем дээр суурилсан чатбот”сэдвийн хүрээнд судалгаа хийхдээ чатбот системийн талаар болон машин сургалт, өгөгдөл цуглуулгын аргын талаар судалсан. Үүний дараа ижил төстэй системийн болон, ашиглагдах технологийн талаар судалгааг хийсэн болно.
\subsection{Чатбот систем}
Чатбот систем нь ихэвчлэн хэрэглэгчийн асуултыг хиймэл оюун ухааны тусламжтайгаар ойлгож, хариултыг автоматжуулах үндсэн зорилготой компьютерийн програм хангамж юм. Орчин үед хэрэглэгчдэд туслах үндсэн үүргийн дагуу чатбот системийг байгууллагууд олон янзаар ашиглах болсон. Тэдгээрээс дурдвал,
\begin{itemize}
  \item Цэс дээр суурилсан чатбот (Menu-based chatbot)
  \item Түлхүүр үгийг танихад суурилсан чатбот (Keyword recognition-based chatbot)
  \item Машин сургалтын чатбот (Machine learning chatbot)
\end{itemize}
\textbf{Цэс дээр суурилсан чатбот}
\\Өнөөгийн зах зээлд хэрэгжиж буй чатботуудын хамгийн энгийн бөгөөд түгээмэл хэлбэр юм.\cite{chatbotsystem} \footnote{\url{https://www.engati.com/blog/types-of-chatbots-and-their-applications}} Хэрэглэгчийн асууж болох асуултуудыг урьдаас таамаглан хариултуудыг мод хэлбэртэйгээр бүтэцлэн хадгалдаг. Хэрэглэгч хүссэн хариултаа авахын тулд системийн хадгалсан хариултаар аялах хэрэгтэй болдог. Бусад чатботтой харьцуулбал, хариулт хязгаарлагдмал бөгөөд хэрэглэгчээс олон асуулт асууж цаг их шаарддагаараа сул талтай байдаг. 
\\
\textbf{Түлхүүр үгийг танихад суурилсан чатбот}
\\Энэхүү чатбот нь хэрэглэгчийн бичсэнийг уншиж тохиромжтой хариултыг өгдөг. Ингэхдээ өгүүлбэрийг хиймэл оюун ухааны нэг хэсэг болох эх хэлний боловсруулалт (Natural Language Processing)-ын тусламжтайгаар шинжилж түлхүүр үгийг таньж хариултыг өгдөг. Ижил төстэй олон асуултад хариулах эсвэл түлхүүр үг дутуу үед амжилтгүй болдог. Мөн хэрэглэгч хүссэн хариултаа олж чадахгүй байх болон үр дүн муутай хариулт өгсөн тохиолдолд цэс дээр суурилсан чатботыг хослуулан ашиглах нь найдвартай болдог бөгөөд түгээмэл шийдлүүдийн нэг байдаг. 
\\
\textbf{Машин сургалтын чатбот}
\\Энэ төрлийн чатбот нь өмнө хэрэглэгчийн харилцан яриан дээр хиймэл оюун ухаан болон машин сургалтын тусламжтайгаар шинжилгээ хийж, хэрэглэгчийн зан төлөв, асуултын хэв маягаас суралцдаг. Ингэснээрээ чатботод хэрэглэгчийн зарцуулах цаг эрчимтэйгээр буурах буюу хариултаа авах алхам багасгах ба хэрэглэгчийн туршлага (UX) нь түүнийгээ даган өсөх нь энэхүү чатботын үндсэн зорилго болно. 
\\
\textbf{Системийг сонгох}
\\
\subsection{}
\section{Ижил төстэй системүүд}

\section{Технологийн судалгаа}