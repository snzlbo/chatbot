%----------------------------------------------------------------------------------------
%   Доорх хэсгийг өөрчлөх шаардлагагүй
%----------------------------------------------------------------------------------------
%!TEX TS-program = xelatex
%!TEX encoding = UTF-8 Unicode
\documentclass[12pt,A4]{report}

\usepackage{fontspec,xltxtra,xunicode}
\setmainfont[Ligatures=TeX]{Times New Roman}
\setsansfont{Arial}

% \usepackage[utf8x]{inputenc}
% \usepackage[mongolian]{babel}
%\usepackage{natbib}
\usepackage{geometry}
%\usepackage{fancyheadings} fancyheadings is obsolete: replaced by fancyhdr. JL
\usepackage{fancyhdr}
\usepackage{float}
\usepackage{afterpage}
\usepackage{graphicx}
\usepackage{amsmath,amssymb,amsbsy}
\usepackage{dcolumn,array}
\usepackage{tocloft}
\usepackage{dics}
\usepackage{nomencl}
\usepackage{upgreek}
\newcommand{\argmin}{\arg\!\min}
\usepackage{mathtools}
\usepackage[hidelinks]{hyperref}
\usepackage{tabularx}

\usepackage{algorithm}
\usepackage{algpseudocode}

\usepackage{listings}
\DeclarePairedDelimiter\abs{\lvert}{\rvert}%
\makeatletter
\usepackage{caption}
\captionsetup[table]{belowskip=0.5pt}
\usepackage{subfiles}

\usepackage{listings}
\renewcommand{\lstlistingname}{Код}
\renewcommand{\lstlistlistingname}{\lstlistingname ын жагсаалт}

\usepackage{color}
\definecolor{codegreen}{rgb}{0,0.6,0}
\definecolor{codegray}{rgb}{0.5,0.5,0.5}
\definecolor{codepurple}{rgb}{0.58,0,0.82}
\definecolor{backcolour}{rgb}{0.99,0.99,0.99}
 
\lstdefinestyle{mystyle}{
    basicstyle=\ttfamily\small,
    backgroundcolor=\color{backcolour},   
    commentstyle=\color{codegreen},
    keywordstyle=\color{magenta},
    numberstyle=\tiny\color{codegray},
    stringstyle=\color{codepurple},
    %basicstyle=\footnotesize,
    breakatwhitespace=false,         
    breaklines=true,                 
    captionpos=b,                    
    keepspaces=false,                 
    numbers=left,                    
    numbersep=10pt,                  
    showspaces=false,                
    showstringspaces=true,
    showtabs=false,                  
    tabsize=2
}
 
\lstset{style=mystyle, label=DescriptiveLabel} 

\let\oldabs\abs
\def\abs{\@ifstar{\oldabs}{\oldabs*}}
\makenomenclature
\begin{document}


%----------------------------------------------------------------------------------------
%   Өөрийн мэдээллээ оруулах хэсэг
%----------------------------------------------------------------------------------------

% Дипломийн ажлын сэдэв
\title{АЖИЛ ОЛГОГЧДЫН ӨГӨГДЛИЙН АНАЛИЗ СИСТЕМ ДЭЭР СУУРИЛСАН ЧАТ БОТ}
% Дипломын ажлын англи нэр
\titleEng{Chat bot based on sytem analysis of employers' data}
% Өөрийн овог нэрийг бүтнээр нь бичнэ
\author{Анужингийн Сайнзолбоо}
% Өөрийн овгийн эхний үсэг нэрээ бичнэ
\authorShort{А.Сайнзолбоо}
% Удирдагчийн зэрэг цол овгийн эхний үсэг нэр
\supervisor{Б.Хуягбаатар доктор (Ph.D.)}
% Хамтарсан удирдагчийн зэрэг цол овгийн эхний үсэг нэр

% СиСи дугаар 
\sisiId{18B1NUM1762}
% Их сургуулийн нэр
\university{МОНГОЛ УЛСЫН ИХ СУРГУУЛЬ}
% Бүрэлдэхүүн сургуулийн нэр
\faculty{ХЭРЭГЛЭЭНИЙ ШИНЖЛЭХ УХААН, ИНЖЕНЕРЧЛЭЛИЙН СУРГУУЛЬ}
% Тэнхимийн нэр
\department{МЭДЭЭЛЭЛ, КОМПЬЮТЕРИЙН УХААНЫ ТЭНХИМ}
% Зэргийн нэр
\degreeName{Бакалаврын судалгааны ажил}
% Суралцаж буй хөтөлбөрийн нэр
\programeName{Мэдээллийн технологи (D061303)}
% Хэвлэгдсэн газар
\cityName{Улаанбаатар}
% Хэвлэгдсэн огноо
\gradyear{2022 оны 03 сар}


%----------------------------------------------------------------------------------------
%   Доорх хэсгийг өөрчлөх шаардлагагүй
%----------------------------------------------------------------------------------------
%----------------------Нүүр хуудастай хамаатай зүйлс----------------------------
\pagenumbering{roman}
\makefrontpage
\maketitle

\doublespace

% Гарчгийг автоматаар оруулна
\setcounter{tocdepth}{1}
\tableofcontents

% Зургийн жагсаалтыг автоматаар оруулна
\listoffigures

% This puts the word "Page" right justified above everything else.
\newpage
%% \addtocontents{lof}{Зураг~\hfill Хуудас \par}
\newpage
%% \addtocontents{lot}{Хүснэгт~\hfill Хуудас \par}

\renewcommand{\cftlabel}{Зураг}


\doublespace
\pagenumbering{arabic}


\begin{abstract}

Мэдээллийн технологи эрчимтэй хөгжиж буй өнөөгийн нийгэмд байгууллага үйл ажиллагаа явуулж эхэлсэн цагаасаа эхлэн өгөгдлийг үйлдвэрлэсээр байдаг. Тэдгээр өгөгдлийг байнга хадгалах нь өгөгдлийн сангийн нөөцөд хортой байдаг тул өгөгдөлд шинжилгээ хийж, тэдгээрээс шаардлагатай өгөгдлүүдийг түүвэрлэн хадгалах нь чухал юм.

\quad Өнөөдөр бид дэлхий нийтээрээ хурдтай амьдралын хэмнэлд ажиллаж, амьдарч байна. Мөн зах зээлийн хөгжил, ажил олгогчийн эрэлт хэрэгцээ ажил хайгчийн хүсэл онирхлыг оновчтой бөгөөд хурдан холбож өгөх нь нэн шаардлагатай. Өнөөгийн байдлаар энэ эрэлт хэрэгцээг хангасан тодорхой шийдвэрлэсэн мэдээллийн систем хомс байна. Иймд энэхүү бакалаврын судалгааны ажлаар ажил олгогч болон ажил идэвхтэй хайгч хоёрыг түргэн шуурхай холбож өгөх чатбот системийг хөгжүүлж байна. 

\end{abstract}

\addcontentsline{toc}{part}{БҮЛГҮҮД}
\chapter{Сэдвийн танилцуулга}
\subfile{chapters/introduction}

\chapter{Системийн судалгаа}
\subfile{chapters/research.tex}

\chapter{Системийн шинжилгээ}
\subfile{chapters/analysis.tex}

\chapter{Системийн зохиомж}
\subfile{chapters/design.tex}

\chapter{Хэрэгжүүлэлт, үр дүн}
\subfile{chapters/implement.tex}

\addcontentsline{toc}{part}{ДҮГНЭЛТ}

\conclusion{Дүгнэлт}
\quad Бакалаврын судалгааны ажлаар ``Ажил олгогчдын өгөгдлийн анализ систем дээр суурилсан чатбот'' сэдвийн дагуу хөгжүүлэлтийг эхлүүлсэн бөгөөд уг судалгааны ажилд холбогдох онолын судалгаа, ашиглаж буй технологи, түүнийг илүү онолын мэдлэг болон системийн хөгжүүлэлтийн хэсгээс дэлгэрэнгүй тайлбарласан болно.

\singlespace
\addcontentsline{toc}{part}{НОМ ЗҮЙ}
\begin{thebibliography}{99}
	\bibitem{chatbotsystem}
	Чатбот системийн тухай
	\\\url{https://www.engati.com/blog/types-of-chatbots-and-their-applications}

	\bibitem{sentenceTransform}
	Өгүүлбэр хувиргалтын арга зүй
	\\\url{https://www.sbert.net/docs/quickstart.html}
	
	\bibitem{sbert}
	Sentence-BERT: Sentence Embeddings using Siamese BERT-Networks
	\\\url{https://arxiv.org/abs/1908.10084}

	\bibitem{useCase}
	Use case diagram
	\\\url{https://app.diagrams.net/#G1jhom3sc_holt-X9XLALtQja_Gl_Eykhj}

	\bibitem{BPMN}
	Business Process Model Notation 2.0 диаграмм
	\\\url{https://cawemo.com/diagrams/ea037ec0-c1c5-4ab6-8262-521657472803--bpmn-2-0?v=960,418,1}

	\bibitem{dbDiagram}
	Өгөгдлийн сангийн диаграмм
	\\\url{https://dbdiagram.io/d/6249fb7cd043196e39e87451}
\end{thebibliography}



%----------------------------------------------------------------------------------------
%   Хавсралтууд эндээс эхэлнэ
%----------------------------------------------------------------------------------------
\appendix
\addcontentsline{toc}{part}{ХАВСРАЛТ}

% Хавсралтын нэр. Хавсралт гэдэг үг агуулахгүй
\chapter{Үечилсэн төлөвлөгөө}
\begin{figure}[h]
	\centering
	\includegraphics[width=16.5cm, angle=90]{images/plan.png}
	\caption{Бакалаврын судалгааны ажлын үечилсэн төлөвлөгөө}
	\label{fig:plan01}
\end{figure}
% Хавсралтын нэр. Хавсралт гэдэг үг агуулахгүй
\chapter{Кодын хэрэгжүүлэлт}

\section{Өгөгдөл цугуулалт}
Өгөгдөл цуглуулах програм нь дараах бүтэцтэй байх бөгөөд assets доторх кодууд нь үндсэн кодыг ажлуулахад туслах функцууд байна.
\begin{figure}[h]
	\centering
	\includegraphics[width=5cm]{images/folderStructure.png}
	\caption{Фолдерийн бүтэц}
	\label{fig:plan01}
\end{figure}
\subsection{Үндсэн өгөгдлийг цуглуулах эх код}
\lstinputlisting[language=Python, caption=Бүх өгөгдлийг цуглуулах - dataScrapping.py]{code/dataScrapping.py}
\subsection{Нэг зарын шаардлагатай бүх мэдээллийг цуглуулах код}
\lstinputlisting[language=Python, caption=Нэг зарын өгөгдлийг цуглуулах - adScrape.py]{code/adScrape.py}
\subsection{Цуглуулах өгөгдлийн төрөл}
\lstinputlisting[language=Python, caption=Өгөгдлийн төрөл - classTypes.py]{code/classTypes.py}
\subsection{BeautifulSoup scraper}
\lstinputlisting[language=Python, caption=Scrape хийх функц - scrape.py]{code/scrape.py}

\end{document}